\documentclass[handout]{ximera}

\title{The warehouse problem}

\begin{document}
\begin{abstract}
In this activity, we explore a fundamental property of multiplication. 
\end{abstract}
\maketitle

In a warehouse you obtain $20\%$ discount but you must pay a $15\%$
sales tax.

\begin{question}
Which would save you more money: To have the tax calculated first or
the discount?
\begin{solution}
\begin{multiple-choice}
\choice{Apply the tax first.}
\choice{Apply the discount first.} 
\choice[correct]{It doesn't matter.}
\end{multiple-choice}
\end{solution}
\end{question}

Now that you have an answer, let's see if we can figure out why it is
correct. 

\begin{question}
Suppose the item you are interested in buying is 100
dollars. Now suppose the tax is computed first. What is the cost of
the item including tax?
\begin{solution}
\begin{hint}
What is $15\%$ of $100$?  
\end{hint}
The cost of the item and the tax is
$\$ $ \answer{115}.
\end{solution}
 Now take your answer from before (the cost of the item and the tax),
and apply the $20\%$ discount. How much is the total cost now?
\begin{solution}
The total cost is $\$ $ \answer{92}. 
\end{solution}
\end{question}

Now let's compute the total cost computing the discount first. 

\begin{question}
If the item cost $100$ dollars before the discount, what is the cost
before tax?
\begin{solution}
The cost of the item with the discount, but before tax is $\$ $ \answer{80}. 
\end{solution}
 Now if you apply the $15\%$ tax, what is the total cost?
\begin{solution}
The total cost is $\$ $ \answer{92}.
\end{solution} 
\end{question}

\begin{exploration}
What happened? Does this always work? Give a general explanation.
\begin{free-response}
Try different numbers, what property of multiplication are we
utilizing here?
\end{free-response}
\end{exploration}

\end{document}
