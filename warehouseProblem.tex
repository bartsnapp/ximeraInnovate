\documentclass{ximera}

\title{The warehouse problem}

\begin{document}
\begin{abstract}
In this activity, we explore a fundamental property of multiplication. 
\end{abstract}
\maketitle

\begin{question}
In a warehouse you obtain $20\%$ discount but you must pay a $15\%$
sales tax. 

\begin{enumerate}
\item Which would save you more money: To have the tax calculated
  first or the discount?
\begin{solution}
\begin{free-response}
Just give your best guess based on your experience in life. 
\end{free-response}
\end{solution}
\item Now that you have an answer, let's see if we can figure out if
  it is correct or not. Suppose the item you are interested in buying
  is 100 dollars. Now suppose the tax is computed first. What is the
  cost of the item including tax?
\begin{solution}
\begin{hint}
What is $15\%$ of $100$?  
\end{hint}
The cost of the item and the tax is
\answer{115} dollars. 
\end{solution}
\item Now take your answer from before (the cost of the item and the tax),
and apply the $20\%$ discount. How much is the total cost now?
\begin{solution}
The total cost is \answer{92} dollars. 
\end{solution}
\item Now let's compute the total cost computing the discount first. If the
item cost $100$ dollars before the discount, what is the cost before tax?
\begin{solution}
The cost of the item with the discount, but before tax is \answer{80} dollars. 
\end{solution}
\item Now if you apply the $15\%$ tax, what is the total cost?
\begin{solution}
The total cost is \answer{92} dollars.
\end{solution} 
\item What happened? Does this always work? Give a general explanation.
\begin{solution}
\begin{free-response}
Try different numbers, what property of multiplication are we
utilizing here?
\end{free-response}
\end{solution}
\end{enumerate}
\end{question}

\end{document}
